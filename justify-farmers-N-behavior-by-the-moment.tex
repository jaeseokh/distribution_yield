% Options for packages loaded elsewhere
\PassOptionsToPackage{unicode}{hyperref}
\PassOptionsToPackage{hyphens}{url}
\PassOptionsToPackage{dvipsnames,svgnames,x11names}{xcolor}
%
\documentclass[
]{article}

\usepackage{amsmath,amssymb}
\usepackage{iftex}
\ifPDFTeX
  \usepackage[T1]{fontenc}
  \usepackage[utf8]{inputenc}
  \usepackage{textcomp} % provide euro and other symbols
\else % if luatex or xetex
  \usepackage{unicode-math}
  \defaultfontfeatures{Scale=MatchLowercase}
  \defaultfontfeatures[\rmfamily]{Ligatures=TeX,Scale=1}
\fi
\usepackage[]{times}
\ifPDFTeX\else  
    % xetex/luatex font selection
\fi
% Use upquote if available, for straight quotes in verbatim environments
\IfFileExists{upquote.sty}{\usepackage{upquote}}{}
\IfFileExists{microtype.sty}{% use microtype if available
  \usepackage[]{microtype}
  \UseMicrotypeSet[protrusion]{basicmath} % disable protrusion for tt fonts
}{}
\makeatletter
\@ifundefined{KOMAClassName}{% if non-KOMA class
  \IfFileExists{parskip.sty}{%
    \usepackage{parskip}
  }{% else
    \setlength{\parindent}{0pt}
    \setlength{\parskip}{6pt plus 2pt minus 1pt}}
}{% if KOMA class
  \KOMAoptions{parskip=half}}
\makeatother
\usepackage{xcolor}
\setlength{\emergencystretch}{3em} % prevent overfull lines
\setcounter{secnumdepth}{2}
% Make \paragraph and \subparagraph free-standing
\makeatletter
\ifx\paragraph\undefined\else
  \let\oldparagraph\paragraph
  \renewcommand{\paragraph}{
    \@ifstar
      \xxxParagraphStar
      \xxxParagraphNoStar
  }
  \newcommand{\xxxParagraphStar}[1]{\oldparagraph*{#1}\mbox{}}
  \newcommand{\xxxParagraphNoStar}[1]{\oldparagraph{#1}\mbox{}}
\fi
\ifx\subparagraph\undefined\else
  \let\oldsubparagraph\subparagraph
  \renewcommand{\subparagraph}{
    \@ifstar
      \xxxSubParagraphStar
      \xxxSubParagraphNoStar
  }
  \newcommand{\xxxSubParagraphStar}[1]{\oldsubparagraph*{#1}\mbox{}}
  \newcommand{\xxxSubParagraphNoStar}[1]{\oldsubparagraph{#1}\mbox{}}
\fi
\makeatother


\providecommand{\tightlist}{%
  \setlength{\itemsep}{0pt}\setlength{\parskip}{0pt}}\usepackage{longtable,booktabs,array}
\usepackage{calc} % for calculating minipage widths
% Correct order of tables after \paragraph or \subparagraph
\usepackage{etoolbox}
\makeatletter
\patchcmd\longtable{\par}{\if@noskipsec\mbox{}\fi\par}{}{}
\makeatother
% Allow footnotes in longtable head/foot
\IfFileExists{footnotehyper.sty}{\usepackage{footnotehyper}}{\usepackage{footnote}}
\makesavenoteenv{longtable}
\usepackage{graphicx}
\makeatletter
\def\maxwidth{\ifdim\Gin@nat@width>\linewidth\linewidth\else\Gin@nat@width\fi}
\def\maxheight{\ifdim\Gin@nat@height>\textheight\textheight\else\Gin@nat@height\fi}
\makeatother
% Scale images if necessary, so that they will not overflow the page
% margins by default, and it is still possible to overwrite the defaults
% using explicit options in \includegraphics[width, height, ...]{}
\setkeys{Gin}{width=\maxwidth,height=\maxheight,keepaspectratio}
% Set default figure placement to htbp
\makeatletter
\def\fps@figure{htbp}
\makeatother

% Place figures and tables exactly where they were called
\usepackage{float}
\floatplacement{figure}{H}
\floatplacement{table}{H}

% Recommended by the modelsummary package
\usepackage{booktabs}
\usepackage{siunitx}
\newcolumntype{d}{S[input-symbols = ()]}

% Add affiliations (title.tex needs to be called under template-partials)
\usepackage[noblocks]{authblk}
\renewcommand*{\Authsep}{, }
\renewcommand*{\Authand}{, }
\renewcommand*{\Authands}{, }
\renewcommand\Affilfont{\small}

% Add line numbers
\usepackage{lineno}
\linenumbers
\makeatletter
\@ifpackageloaded{caption}{}{\usepackage{caption}}
\AtBeginDocument{%
\ifdefined\contentsname
  \renewcommand*\contentsname{Table of contents}
\else
  \newcommand\contentsname{Table of contents}
\fi
\ifdefined\listfigurename
  \renewcommand*\listfigurename{List of Figures}
\else
  \newcommand\listfigurename{List of Figures}
\fi
\ifdefined\listtablename
  \renewcommand*\listtablename{List of Tables}
\else
  \newcommand\listtablename{List of Tables}
\fi
\ifdefined\figurename
  \renewcommand*\figurename{Figure}
\else
  \newcommand\figurename{Figure}
\fi
\ifdefined\tablename
  \renewcommand*\tablename{Table}
\else
  \newcommand\tablename{Table}
\fi
}
\@ifpackageloaded{float}{}{\usepackage{float}}
\floatstyle{ruled}
\@ifundefined{c@chapter}{\newfloat{codelisting}{h}{lop}}{\newfloat{codelisting}{h}{lop}[chapter]}
\floatname{codelisting}{Listing}
\newcommand*\listoflistings{\listof{codelisting}{List of Listings}}
\makeatother
\makeatletter
\makeatother
\makeatletter
\@ifpackageloaded{caption}{}{\usepackage{caption}}
\@ifpackageloaded{subcaption}{}{\usepackage{subcaption}}
\makeatother

\ifLuaTeX
  \usepackage{selnolig}  % disable illegal ligatures
\fi
\usepackage{bookmark}

\IfFileExists{xurl.sty}{\usepackage{xurl}}{} % add URL line breaks if available
\urlstyle{same} % disable monospaced font for URLs
\hypersetup{
  pdftitle={Dissertation Chapter: N rate justification, higher moment of yield, cliamate uncertainty},
  pdfkeywords={Distribution of moment, Need it, Yield N Response},
  colorlinks=true,
  linkcolor={blue},
  filecolor={Maroon},
  citecolor={red},
  urlcolor={Blue},
  pdfcreator={LaTeX via pandoc}}


\title{Dissertation Chapter: N rate justification, higher moment of
yield, cliamate uncertainty}



\date{2024-11-07}
\begin{document}
\maketitle
\begin{abstract}
Start with literature review and idea about how to estimate moment
function with DIFM data
\end{abstract}


\begin{center}\rule{0.5\linewidth}{0.5pt}\end{center}

\section{Dissertation Outline}\label{dissertation-outline}

\subsection{1. Introduction}\label{introduction}

In recent years, the economic and environmental concerns associated with
nitrogen (N) fertilizer application in agriculture have gained
increasing attention. Despite considerable progress in estimating
economically optimal nitrogen rates (EONR), empirical evidence suggests
that farmers often apply nitrogen rates exceeding EONR. This research
aims to evaluate the appropriateness of farmers' nitrogen application
decisions by examining the yield distribution's higher moments
(variance, skewness, and kurtosis) under different climate and nitrogen
interactions.

\subsubsection{Research Objectives}\label{research-objectives}

\begin{itemize}
\tightlist
\item
  To determine if observed nitrogen application rates by farmers exceed
  the estimated EONR.
\item
  To evaluate the impact of nitrogen rates and climate factors on the
  \textbf{higher-order moments} (variance, skewness, kurtosis) of yield
  distribution.
\item
  To explore whether the observed over-application of nitrogen by
  farmers is justified in terms of profitability, taking into account
  climate uncertainty and price volatility.
\end{itemize}

The methods used are built upon Antle (1983) and Tack, Harri, and Coble
(2012), focusing on how higher-order moments of yield distribution can
be influenced by climate and input interactions.

\subsection{2. Literature Review}\label{literature-review}

The literature review focuses on the relationship between nitrogen use,
yield response, and climate interactions.

\subsubsection{2.1. Optimal Nitrogen
Application}\label{optimal-nitrogen-application}

\begin{itemize}
\tightlist
\item
  \textbf{Economically Optimal Nitrogen Rates (EONR)}: Discuss studies
  that have estimated the EONR and provide insights into how observed
  farmer behavior deviates from optimal levels. Emphasize how
  over-application leads to diminishing returns and environmental harm.
\end{itemize}

\subsubsection{2.2. Yield Distributions and
Moments}\label{yield-distributions-and-moments}

\begin{itemize}
\tightlist
\item
  \textbf{Antle (1983)}: Introduced the concept of modeling yield
  distributions based on higher-order moments, using \textbf{production
  moments} to represent stochastic production functions. Farmers'
  preferences over different moments (mean, variance, skewness) can
  determine their risk management strategies.
\item
  \textbf{Tack, Harri, and Coble (2012)}: Extended Antle's work by
  examining the impact of climate on higher-order yield moments,
  employing a \textbf{maximum entropy approach} to derive yield
  distributions under uncertainty.
\end{itemize}

\subsubsection{2.3. Nitrogen Use and Climate
Uncertainty}\label{nitrogen-use-and-climate-uncertainty}

\begin{itemize}
\tightlist
\item
  Explore literature on how \textbf{climate variability} (e.g.,
  temperature, precipitation) interacts with nitrogen application to
  impact yield. Include studies that consider \textbf{ex-ante} versus
  \textbf{ex-post} perspectives of uncertainty.
\end{itemize}

\subsection{3. Methodology}\label{methodology}

\subsubsection{3.1. Data}\label{data}

\begin{itemize}
\tightlist
\item
  \textbf{Field Experiment Data}: Utilize on-farm experimental data from
  100 fields with randomized nitrogen and seeding rate applications.
\item
  \textbf{Weather and Soil Data}: Include weather data (e.g., rainfall,
  temperature) and soil characteristics for each field.
\end{itemize}

\subsubsection{3.2. Econometric Model for EONR
Estimation}\label{econometric-model-for-eonr-estimation}

\begin{itemize}
\item
  \textbf{Regression Analysis}: Estimate yield response as a function of
  nitrogen rate (\(N\)), seeding rate (\(S\)), and control variables
  (soil characteristics, weather). \[
  Y_i =eta_0 +eta_1 N_i + eta_2 S_i + \sum_{j=1}^{K} \gamma_j X_{ij} + \epsilon_i
  \] where \(Y_i\) is the yield, \(N_i\) is nitrogen rate, \(S_i\) is
  seeding rate, and \(X_{ij}\) represents other field-specific variables
  (e.g., climate, soil).
\item
  \textbf{Test for Over-application}: Compare observed farmer nitrogen
  application rates (\(N_{   ext{farmer}}\)) to the estimated EONR
  (\(N_{  ext{EONR}}\)). \[
  H_0: N_{  ext{farmer}} = N_{  ext{EONR}} \quad \text{vs.} \quad H_1: N_{  ext{farmer}} > N_{  ext{EONR}}
  \]
\end{itemize}

\subsubsection{3.3. Modeling Higher Moments of Yield
Distribution}\label{modeling-higher-moments-of-yield-distribution}

\begin{itemize}
\item
  \textbf{Moment-based Yield Response}: Adapt Antle (1983)'s approach to
  model the \textbf{second} (\(\mu_2\)), \textbf{third} (\(\mu_3\)), and
  \textbf{fourth} (\(\mu_4\)) moments of yield. \[
  \mu_1(N, Z) = E[Y | N, Z]
  \] \[
  \mu_2(N, Z) = E[(Y - \mu_1)^2 | N, Z]
  \] \[
  \mu_3(N, Z) = E[(Y - \mu_1)^3 | N, Z]
  \] \[
  \mu_4(N, Z) = E[(Y - \mu_1)^4 | N, Z]
  \] where \(Z\) represents climate variables.
\item
  \textbf{Maximum Entropy Approach}: Following Tack et al.~(2012),
  estimate the yield distribution under nitrogen-climate interaction by
  maximizing entropy, subject to moment constraints. \[
  f^*(Y) = \frac{1}{\psi(\gamma^*)} \exp \left( - \sum_{j=1}^J \gamma_j^* Y^j \right)
  \] where \(\gamma_j^*\) are the Lagrange multipliers obtained from
  moment conditions, and \(\psi(\gamma^*)\) is a normalization factor.
\end{itemize}

\subsection{4. Results and Discussion}\label{results-and-discussion}

\subsubsection{4.1. Estimation of EONR}\label{estimation-of-eonr}

\begin{itemize}
\tightlist
\item
  Present the results of the yield response regression model.
\item
  Discuss the extent to which observed nitrogen rates exceed the
  estimated EONR and potential reasons for this behavior (e.g., risk
  aversion, yield variability).
\end{itemize}

\subsubsection{4.2. Impact of Nitrogen and Climate on Yield
Moments}\label{impact-of-nitrogen-and-climate-on-yield-moments}

\begin{itemize}
\tightlist
\item
  \textbf{Variance (\(\mu_2\))}: Analyze how nitrogen and climate
  jointly impact yield variability. High variance may indicate that
  nitrogen over-application leads to unpredictable outcomes under
  climate variability.
\item
  \textbf{Skewness (\(\mu_3\))}: Evaluate whether the distribution of
  yield outcomes is \textbf{negatively skewed}, implying higher risk of
  low yields under excess nitrogen application.
\item
  \textbf{Kurtosis (\(\mu_4\))}: Assess if higher kurtosis suggests
  greater probability of extreme yield outcomes (both high and low).
\end{itemize}

\subsubsection{4.3. Implications for Farmer
Behavior}\label{implications-for-farmer-behavior}

\begin{itemize}
\tightlist
\item
  Discuss the implications of yield variability and risk, relating back
  to farmers' over-application behavior.
\item
  Argue that, given the observed patterns in higher moments,
  over-application of nitrogen is \textbf{not economically justified}
  due to increased risk and lack of consistent yield gains under varying
  climate conditions.
\end{itemize}

\subsection{5. Conclusion}\label{conclusion}

\begin{itemize}
\tightlist
\item
  Summarize the key findings regarding nitrogen application, yield
  response, and the effect of climate interactions on yield distribution
  moments.
\item
  Highlight policy recommendations for encouraging farmers to optimize
  nitrogen use, considering yield variability and environmental
  sustainability.
\end{itemize}

\subsection{6. References}\label{references}

\begin{itemize}
\tightlist
\item
  Antle, J.M. (1983). \emph{Sequential Moments and the Analysis of Risk
  Preferences.}
\item
  Tack, J., Harri, A., \& Coble, K. (2012). \emph{More than mean
  effects: Modeling the effect of climate on the higher order moments of
  crop yields.} American Journal of Agricultural Economics, 94(5),
  1037-1054.
\end{itemize}




\end{document}
