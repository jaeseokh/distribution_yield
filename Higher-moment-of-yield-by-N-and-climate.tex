% Options for packages loaded elsewhere
\PassOptionsToPackage{unicode}{hyperref}
\PassOptionsToPackage{hyphens}{url}
\PassOptionsToPackage{dvipsnames,svgnames,x11names}{xcolor}
%
\documentclass[
  letterpaper,
  DIV=11,
  numbers=noendperiod]{scrartcl}

\usepackage{amsmath,amssymb}
\usepackage{iftex}
\ifPDFTeX
  \usepackage[T1]{fontenc}
  \usepackage[utf8]{inputenc}
  \usepackage{textcomp} % provide euro and other symbols
\else % if luatex or xetex
  \usepackage{unicode-math}
  \defaultfontfeatures{Scale=MatchLowercase}
  \defaultfontfeatures[\rmfamily]{Ligatures=TeX,Scale=1}
\fi
\usepackage[]{times}
\ifPDFTeX\else  
    % xetex/luatex font selection
\fi
% Use upquote if available, for straight quotes in verbatim environments
\IfFileExists{upquote.sty}{\usepackage{upquote}}{}
\IfFileExists{microtype.sty}{% use microtype if available
  \usepackage[]{microtype}
  \UseMicrotypeSet[protrusion]{basicmath} % disable protrusion for tt fonts
}{}
\makeatletter
\@ifundefined{KOMAClassName}{% if non-KOMA class
  \IfFileExists{parskip.sty}{%
    \usepackage{parskip}
  }{% else
    \setlength{\parindent}{0pt}
    \setlength{\parskip}{6pt plus 2pt minus 1pt}}
}{% if KOMA class
  \KOMAoptions{parskip=half}}
\makeatother
\usepackage{xcolor}
\setlength{\emergencystretch}{3em} % prevent overfull lines
\setcounter{secnumdepth}{2}
% Make \paragraph and \subparagraph free-standing
\makeatletter
\ifx\paragraph\undefined\else
  \let\oldparagraph\paragraph
  \renewcommand{\paragraph}{
    \@ifstar
      \xxxParagraphStar
      \xxxParagraphNoStar
  }
  \newcommand{\xxxParagraphStar}[1]{\oldparagraph*{#1}\mbox{}}
  \newcommand{\xxxParagraphNoStar}[1]{\oldparagraph{#1}\mbox{}}
\fi
\ifx\subparagraph\undefined\else
  \let\oldsubparagraph\subparagraph
  \renewcommand{\subparagraph}{
    \@ifstar
      \xxxSubParagraphStar
      \xxxSubParagraphNoStar
  }
  \newcommand{\xxxSubParagraphStar}[1]{\oldsubparagraph*{#1}\mbox{}}
  \newcommand{\xxxSubParagraphNoStar}[1]{\oldsubparagraph{#1}\mbox{}}
\fi
\makeatother


\providecommand{\tightlist}{%
  \setlength{\itemsep}{0pt}\setlength{\parskip}{0pt}}\usepackage{longtable,booktabs,array}
\usepackage{calc} % for calculating minipage widths
% Correct order of tables after \paragraph or \subparagraph
\usepackage{etoolbox}
\makeatletter
\patchcmd\longtable{\par}{\if@noskipsec\mbox{}\fi\par}{}{}
\makeatother
% Allow footnotes in longtable head/foot
\IfFileExists{footnotehyper.sty}{\usepackage{footnotehyper}}{\usepackage{footnote}}
\makesavenoteenv{longtable}
\usepackage{graphicx}
\makeatletter
\def\maxwidth{\ifdim\Gin@nat@width>\linewidth\linewidth\else\Gin@nat@width\fi}
\def\maxheight{\ifdim\Gin@nat@height>\textheight\textheight\else\Gin@nat@height\fi}
\makeatother
% Scale images if necessary, so that they will not overflow the page
% margins by default, and it is still possible to overwrite the defaults
% using explicit options in \includegraphics[width, height, ...]{}
\setkeys{Gin}{width=\maxwidth,height=\maxheight,keepaspectratio}
% Set default figure placement to htbp
\makeatletter
\def\fps@figure{htbp}
\makeatother

\KOMAoption{captions}{tableheading}
\makeatletter
\@ifpackageloaded{caption}{}{\usepackage{caption}}
\AtBeginDocument{%
\ifdefined\contentsname
  \renewcommand*\contentsname{Table of contents}
\else
  \newcommand\contentsname{Table of contents}
\fi
\ifdefined\listfigurename
  \renewcommand*\listfigurename{List of Figures}
\else
  \newcommand\listfigurename{List of Figures}
\fi
\ifdefined\listtablename
  \renewcommand*\listtablename{List of Tables}
\else
  \newcommand\listtablename{List of Tables}
\fi
\ifdefined\figurename
  \renewcommand*\figurename{Figure}
\else
  \newcommand\figurename{Figure}
\fi
\ifdefined\tablename
  \renewcommand*\tablename{Table}
\else
  \newcommand\tablename{Table}
\fi
}
\@ifpackageloaded{float}{}{\usepackage{float}}
\floatstyle{ruled}
\@ifundefined{c@chapter}{\newfloat{codelisting}{h}{lop}}{\newfloat{codelisting}{h}{lop}[chapter]}
\floatname{codelisting}{Listing}
\newcommand*\listoflistings{\listof{codelisting}{List of Listings}}
\makeatother
\makeatletter
\makeatother
\makeatletter
\@ifpackageloaded{caption}{}{\usepackage{caption}}
\@ifpackageloaded{subcaption}{}{\usepackage{subcaption}}
\makeatother

\ifLuaTeX
  \usepackage{selnolig}  % disable illegal ligatures
\fi
\usepackage{bookmark}

\IfFileExists{xurl.sty}{\usepackage{xurl}}{} % add URL line breaks if available
\urlstyle{same} % disable monospaced font for URLs
\hypersetup{
  pdftitle={Higher Moment of Yield by N x Climate: Definition and Estimation},
  pdfkeywords={Distribution of moment, Yield N Response},
  colorlinks=true,
  linkcolor={blue},
  filecolor={Maroon},
  citecolor={red},
  urlcolor={Blue},
  pdfcreator={LaTeX via pandoc}}


\title{Higher Moment of Yield by N x Climate: Definition and Estimation}
\author{}
\date{2024-11-12}

\begin{document}
\maketitle
\begin{abstract}
This paper explores higher-moments of yield responses to nitrogen under
variable climate conditions. We model the higher moments of yield
distributions using a flexible functional approach.
\end{abstract}


\newpage

\section{Introduction}\label{introduction}

Accurate nitrogen (\(N\)) use is crucial for maximizing crop yields and
profitability under variable climate conditions. Understanding how \(N\)
affects yield distribution is essential for agricultural policy,
especially under climate variability. Traditional analysis focuses on
mean yield, but this overlooks yield variability and higher-order
moments, which are critical for understanding the risk associated with
nitrogen application.

\section{Literature Review}\label{literature-review}

\subsection{Higher Moments of Yield}\label{higher-moments-of-yield}

Studies in agricultural economics show that focusing on only the mean
yield effect is limiting. Literature on higher-order moments (e.g.,
variance, skewness) suggests that these moments provide insights into
yield risk, crucial for risk-averse decision-making (Antle, 1983).

\subsection{Crop Yield Insurance and Nitrogen
Use}\label{crop-yield-insurance-and-nitrogen-use}

In crop insurance, nitrogen is often considered as a factor in yield
stability. Literature indicates that nitrogen can either mitigate or
amplify risks associated with weather variability, with direct
implications for insurance schemes.

\section{Research Objectives}\label{research-objectives}

The primary objective of this research is to evaluate whether \(N\)
application is yield-increasing or yield-decreasing under different
climate conditions, particularly focusing on how \(N\) affects not only
mean yield but also the variance and higher moments of yield under the
different weather events.

\section{Methodology}\label{methodology}

\subsection{Data Structuring and Panel Data
Creation}\label{data-structuring-and-panel-data-creation}

Our dataset consists of approximately 100 on-farm trials with variable
nitrogen rates, seeding rates, and climate conditions.

\subsubsection{Potential problem of making combined panel
data}\label{potential-problem-of-making-combined-panel-data}

\begin{itemize}
\item
  Heterogeneity: Differences in soil, climate, and management practices
  across trials.
\item
  Non-Panel Structure: This dataset is not inherently structured as a
  panel, which poses challenges.
\item
  Solution: Using fixed effects to control for unobserved heterogeneity.
  Additionally, create a pseudo-panel by aggregating trial-level data
  into consistent nitrogen, climate, and yield variables.
\end{itemize}

\subsection{Yield Response and Moment
Estimation}\label{yield-response-and-moment-estimation}

\begin{itemize}
\item
  Estimate yield response to nitrogen using machine learning models to
  capture non-linear relationships.
\item
  Estimate moments ( first and higher )
  \begin{equation}\phantomsection\label{eq-mean-yield}{
  \mu_1 = E[Q] = m(x,\beta) E[e^u]
  }\end{equation}
\end{itemize}

Here, \(m(x,\beta)\) represents the deterministic part of the yield
response, which depends on inputs like nitrogen (\(x\)), while
\(E[e^u]\) represents the stochastic component.

\begin{itemize}
\tightlist
\item
  \textbf{Nitrogen (N)} and other inputs directly affect the
  deterministic part of the production function.
\item
  The weather component (\(e^u\)) influences variability in yield
  outcomes, reflecting production risk.
\end{itemize}

The variance of yield can be expressed as:

\begin{equation}\phantomsection\label{eq-variance-yield}{
\mu_2 = E[(Q - E(Q))^2] = m(x,\beta)^2 (E[e^{2u}] - E[e^u]^2)
}\end{equation}

\subsection{Expected Output Moments and Their Relationship to
Inputs}\label{expected-output-moments-and-their-relationship-to-inputs}

The moments of the probability distribution of output are represented
by:

\begin{equation}\phantomsection\label{eq-moment1}{
\mu_1(x, \gamma_i) = \int Q f(Q|x) \, dQ
}\end{equation}

\begin{equation}\phantomsection\label{eq-momenti}{
\mu_i(x, \gamma_i) = \int (Q - \mu_1)^i f(Q|x) \, dQ, \quad i > 2
}\end{equation}

\begin{itemize}
\tightlist
\item
  \(\mu_1(x, \gamma_i)\) is the first moment (mean yield), and
  \(\mu_i(x, \gamma_i)\) represents higher moments.
\item
  This approach allows for \textbf{heteroskedasticity} (variance
  depending on \(x\)) and \textbf{heteroskewness} (skewness depending on
  \(x\)), providing a more flexible representation of yield
  distributions under different input conditions.
\end{itemize}

\subsection{Empirical Model for regressino of Yield Moments to inputs
and
weather}\label{empirical-model-for-regressino-of-yield-moments-to-inputs-and-weather}

The empirical model is:

\[
y_{it}^j = \alpha_{ij} + \beta_{j1} \text{low}_{it} + \beta_{j2} \text{S}_{it} + \beta_{j3} \text{N}_{it} + \beta_{j4} P_{it} + \beta_{j5} P^2_{it}+ \beta_{j6} G_{it}^2 + \beta_{j7} \text{N*P}_{it} +
\beta_{j8} t + \beta_{j9} t^2 + \epsilon_{ijt}
\]

\begin{itemize}
\tightlist
\item
  \(\alpha_{ij}\) is a fixed effect for field and moment.
\item
  \(p_{it}\) capture precipitation and its quadratic effect.
\item
  \(t\) and \(t^2\) are time trend variables.
\end{itemize}




\end{document}
