% Options for packages loaded elsewhere
\PassOptionsToPackage{unicode}{hyperref}
\PassOptionsToPackage{hyphens}{url}
\PassOptionsToPackage{dvipsnames,svgnames,x11names}{xcolor}
%
\documentclass[
]{article}

\usepackage{amsmath,amssymb}
\usepackage{iftex}
\ifPDFTeX
  \usepackage[T1]{fontenc}
  \usepackage[utf8]{inputenc}
  \usepackage{textcomp} % provide euro and other symbols
\else % if luatex or xetex
  \usepackage{unicode-math}
  \defaultfontfeatures{Scale=MatchLowercase}
  \defaultfontfeatures[\rmfamily]{Ligatures=TeX,Scale=1}
\fi
\usepackage[]{times}
\ifPDFTeX\else  
    % xetex/luatex font selection
\fi
% Use upquote if available, for straight quotes in verbatim environments
\IfFileExists{upquote.sty}{\usepackage{upquote}}{}
\IfFileExists{microtype.sty}{% use microtype if available
  \usepackage[]{microtype}
  \UseMicrotypeSet[protrusion]{basicmath} % disable protrusion for tt fonts
}{}
\makeatletter
\@ifundefined{KOMAClassName}{% if non-KOMA class
  \IfFileExists{parskip.sty}{%
    \usepackage{parskip}
  }{% else
    \setlength{\parindent}{0pt}
    \setlength{\parskip}{6pt plus 2pt minus 1pt}}
}{% if KOMA class
  \KOMAoptions{parskip=half}}
\makeatother
\usepackage{xcolor}
\setlength{\emergencystretch}{3em} % prevent overfull lines
\setcounter{secnumdepth}{2}
% Make \paragraph and \subparagraph free-standing
\makeatletter
\ifx\paragraph\undefined\else
  \let\oldparagraph\paragraph
  \renewcommand{\paragraph}{
    \@ifstar
      \xxxParagraphStar
      \xxxParagraphNoStar
  }
  \newcommand{\xxxParagraphStar}[1]{\oldparagraph*{#1}\mbox{}}
  \newcommand{\xxxParagraphNoStar}[1]{\oldparagraph{#1}\mbox{}}
\fi
\ifx\subparagraph\undefined\else
  \let\oldsubparagraph\subparagraph
  \renewcommand{\subparagraph}{
    \@ifstar
      \xxxSubParagraphStar
      \xxxSubParagraphNoStar
  }
  \newcommand{\xxxSubParagraphStar}[1]{\oldsubparagraph*{#1}\mbox{}}
  \newcommand{\xxxSubParagraphNoStar}[1]{\oldsubparagraph{#1}\mbox{}}
\fi
\makeatother


\providecommand{\tightlist}{%
  \setlength{\itemsep}{0pt}\setlength{\parskip}{0pt}}\usepackage{longtable,booktabs,array}
\usepackage{calc} % for calculating minipage widths
% Correct order of tables after \paragraph or \subparagraph
\usepackage{etoolbox}
\makeatletter
\patchcmd\longtable{\par}{\if@noskipsec\mbox{}\fi\par}{}{}
\makeatother
% Allow footnotes in longtable head/foot
\IfFileExists{footnotehyper.sty}{\usepackage{footnotehyper}}{\usepackage{footnote}}
\makesavenoteenv{longtable}
\usepackage{graphicx}
\makeatletter
\def\maxwidth{\ifdim\Gin@nat@width>\linewidth\linewidth\else\Gin@nat@width\fi}
\def\maxheight{\ifdim\Gin@nat@height>\textheight\textheight\else\Gin@nat@height\fi}
\makeatother
% Scale images if necessary, so that they will not overflow the page
% margins by default, and it is still possible to overwrite the defaults
% using explicit options in \includegraphics[width, height, ...]{}
\setkeys{Gin}{width=\maxwidth,height=\maxheight,keepaspectratio}
% Set default figure placement to htbp
\makeatletter
\def\fps@figure{htbp}
\makeatother

% Place figures and tables exactly where they were called
\usepackage{float}
\floatplacement{figure}{H}
\floatplacement{table}{H}

% Recommended by the modelsummary package
\usepackage{booktabs}
\usepackage{siunitx}
\newcolumntype{d}{S[input-symbols = ()]}

% Add affiliations (title.tex needs to be called under template-partials)
\usepackage[noblocks]{authblk}
\renewcommand*{\Authsep}{, }
\renewcommand*{\Authand}{, }
\renewcommand*{\Authands}{, }
\renewcommand\Affilfont{\small}

% Add line numbers
\usepackage{lineno}
\linenumbers
\makeatletter
\@ifpackageloaded{caption}{}{\usepackage{caption}}
\AtBeginDocument{%
\ifdefined\contentsname
  \renewcommand*\contentsname{Table of contents}
\else
  \newcommand\contentsname{Table of contents}
\fi
\ifdefined\listfigurename
  \renewcommand*\listfigurename{List of Figures}
\else
  \newcommand\listfigurename{List of Figures}
\fi
\ifdefined\listtablename
  \renewcommand*\listtablename{List of Tables}
\else
  \newcommand\listtablename{List of Tables}
\fi
\ifdefined\figurename
  \renewcommand*\figurename{Figure}
\else
  \newcommand\figurename{Figure}
\fi
\ifdefined\tablename
  \renewcommand*\tablename{Table}
\else
  \newcommand\tablename{Table}
\fi
}
\@ifpackageloaded{float}{}{\usepackage{float}}
\floatstyle{ruled}
\@ifundefined{c@chapter}{\newfloat{codelisting}{h}{lop}}{\newfloat{codelisting}{h}{lop}[chapter]}
\floatname{codelisting}{Listing}
\newcommand*\listoflistings{\listof{codelisting}{List of Listings}}
\makeatother
\makeatletter
\makeatother
\makeatletter
\@ifpackageloaded{caption}{}{\usepackage{caption}}
\@ifpackageloaded{subcaption}{}{\usepackage{subcaption}}
\makeatother

\ifLuaTeX
  \usepackage{selnolig}  % disable illegal ligatures
\fi
\usepackage{bookmark}

\IfFileExists{xurl.sty}{\usepackage{xurl}}{} % add URL line breaks if available
\urlstyle{same} % disable monospaced font for URLs
\hypersetup{
  pdftitle={Stat Summary CheatSheet},
  pdfkeywords={Distribution of moment, Need it, Yield N Response},
  colorlinks=true,
  linkcolor={blue},
  filecolor={Maroon},
  citecolor={red},
  urlcolor={Blue},
  pdfcreator={LaTeX via pandoc}}


\title{Stat Summary CheatSheet}



\date{2024-11-06}
\begin{document}
\maketitle
\begin{abstract}
Start with literature review and idea about how to estimate moment
function with DIFM data
\end{abstract}


\subsection{\texorpdfstring{\textbf{Cheat Sheet 1: Linear and Non-Linear
Regression
Models}}{Cheat Sheet 1: Linear and Non-Linear Regression Models}}\label{cheat-sheet-1-linear-and-non-linear-regression-models}

\subsubsection{\texorpdfstring{\textbf{1. Linear Regression
Model}}{1. Linear Regression Model}}\label{linear-regression-model}

\begin{itemize}
\tightlist
\item
  \textbf{Equation}: \[Y = X\beta + \epsilon\] Where \(Y\) is the
  dependent variable, \(X\) is the matrix of independent variables,
  \(\beta\) is the coefficient vector, and \(\epsilon\) represents the
  error term.
\item
  \textbf{Assumptions}:

  \begin{enumerate}
  \def\labelenumi{\arabic{enumi}.}
  \tightlist
  \item
    \textbf{Linearity}: The relationship between \(Y\) and \(X\) is
    linear.
  \item
    \textbf{Full Rank}: The \(X\) matrix has full rank;
    multicollinearity is absent.
  \item
    \textbf{No Endogeneity}: \(X\) and \(\epsilon\) are uncorrelated.
  \item
    \textbf{Homoscedasticity}: Constant variance of the error terms.
  \item
    \textbf{No Autocorrelation}: Errors are not correlated with one
    another.
  \item
    \textbf{Normality of Errors}: Errors are normally distributed for
    inference.
  \end{enumerate}
\item
  \textbf{Violation Impacts}:

  \begin{itemize}
  \tightlist
  \item
    \textbf{Multicollinearity}: Leads to large standard errors for
    \(\beta\), making coefficients imprecise.
  \item
    \textbf{Endogeneity}: Causes bias in \(\beta\) estimates.
  \item
    \textbf{Heteroscedasticity}: Leads to inefficient estimators;
    standard errors are incorrect, affecting hypothesis tests.
  \item
    \textbf{Autocorrelation}: Leads to inefficient \(\beta\) estimates
    and unreliable standard errors.
  \end{itemize}
\item
  \textbf{Remedies}:

  \begin{itemize}
  \tightlist
  \item
    \textbf{Multicollinearity}: Drop collinear variables or use
    regularization techniques (e.g., Ridge/Lasso).
  \item
    \textbf{Endogeneity}: Use instrumental variables (IV).
  \item
    \textbf{Heteroscedasticity}: Use robust standard errors or GLS.
  \item
    \textbf{Autocorrelation}: Use GLS or Newey-West standard errors.
  \end{itemize}
\end{itemize}

\subsubsection{\texorpdfstring{\textbf{2. Non-Linear Regression
Model}}{2. Non-Linear Regression Model}}\label{non-linear-regression-model}

\begin{itemize}
\tightlist
\item
  \textbf{Equation (Example - Logistic Regression)}:
  \[P(Y=1|X) = \frac{1}{1+e^{-X\beta}}\] The response variable is
  binary, and the model is nonlinear in parameters.
\item
  \textbf{Key Assumptions}:

  \begin{itemize}
  \tightlist
  \item
    \textbf{Independent Errors}: Observations are independent.
  \item
    \textbf{Correct Model Specification}: The functional form is
    correctly specified.
  \end{itemize}
\item
  \textbf{Violation Impacts}:

  \begin{itemize}
  \tightlist
  \item
    \textbf{Misspecification}: Leads to biased estimates.
  \item
    \textbf{Multicollinearity}: Impacts the stability of estimated
    coefficients.
  \end{itemize}
\item
  \textbf{Remedies}:

  \begin{itemize}
  \tightlist
  \item
    \textbf{Misspecification}: Use non-parametric techniques to verify
    functional form.
  \item
    \textbf{Multicollinearity}: Use variable selection or
    regularization.
  \end{itemize}
\end{itemize}

\subsubsection{\texorpdfstring{\textbf{3. Bias and
Efficiency}}{3. Bias and Efficiency}}\label{bias-and-efficiency}

\begin{itemize}
\tightlist
\item
  \textbf{Unbiased Estimator}: An estimator is unbiased if
  \(E(\hat{\beta}) = \beta\). Violations like omitted variables or
  endogeneity cause bias.
\item
  \textbf{Efficiency}: An efficient estimator has the smallest variance
  among all unbiased estimators. Violations of homoscedasticity or
  autocorrelation typically lead to inefficiencies.
\end{itemize}

\subsection{\texorpdfstring{\textbf{Cheat Sheet 2: Statistical Tests for
Regression
Models}}{Cheat Sheet 2: Statistical Tests for Regression Models}}\label{cheat-sheet-2-statistical-tests-for-regression-models}

\subsubsection{\texorpdfstring{\textbf{1. Assumption Checks for Linear
Regression}}{1. Assumption Checks for Linear Regression}}\label{assumption-checks-for-linear-regression}

\begin{itemize}
\tightlist
\item
  \textbf{Multicollinearity}:

  \begin{itemize}
  \tightlist
  \item
    \textbf{Variance Inflation Factor (VIF)}: High VIF (\textgreater{}
    10) indicates multicollinearity.
  \end{itemize}
\item
  \textbf{Homoscedasticity}:

  \begin{itemize}
  \tightlist
  \item
    \textbf{Breusch-Pagan Test}: Tests if variance of errors is
    constant.
  \item
    \textbf{White Test}: Tests for heteroscedasticity without assuming a
    specific form.
  \end{itemize}
\item
  \textbf{Normality of Errors}:

  \begin{itemize}
  \tightlist
  \item
    \textbf{Shapiro-Wilk Test}: Tests normality of residuals.
  \item
    \textbf{Q-Q Plot}: Visual inspection for normality.
  \end{itemize}
\item
  \textbf{No Autocorrelation}:

  \begin{itemize}
  \tightlist
  \item
    \textbf{Durbin-Watson Test}: Checks for first-order autocorrelation
    in residuals.
  \end{itemize}
\end{itemize}

\subsubsection{\texorpdfstring{\textbf{2. Assumption Checks for
Non-Linear
Models}}{2. Assumption Checks for Non-Linear Models}}\label{assumption-checks-for-non-linear-models}

\begin{itemize}
\tightlist
\item
  \textbf{Model Fit}:

  \begin{itemize}
  \tightlist
  \item
    \textbf{Likelihood Ratio Test}: Compares nested models to determine
    if added complexity improves fit.
  \item
    \textbf{Wald Test}: Tests the significance of individual regression
    coefficients.
  \end{itemize}
\item
  \textbf{Multicollinearity}:

  \begin{itemize}
  \tightlist
  \item
    \textbf{Condition Index}: High values (\textgreater{} 30) indicate
    multicollinearity.
  \end{itemize}
\item
  \textbf{Goodness of Fit}:

  \begin{itemize}
  \tightlist
  \item
    \textbf{Pseudo \(R^2\) (e.g., McFadden's \(R^2\))}: Used for
    logistic regression to measure model fit.
  \end{itemize}
\end{itemize}

\subsubsection{\texorpdfstring{\textbf{3. Model Feature
Tests}}{3. Model Feature Tests}}\label{model-feature-tests}

\begin{itemize}
\tightlist
\item
  \textbf{Endogeneity}:

  \begin{itemize}
  \tightlist
  \item
    \textbf{Hausman Test}: Compares IV and OLS to determine if an
    endogeneity problem exists.
  \end{itemize}
\item
  \textbf{Nonlinearity}:

  \begin{itemize}
  \tightlist
  \item
    \textbf{RESET Test}: Tests if non-linear combinations of the fitted
    values help explain the response variable.
  \end{itemize}
\end{itemize}

\subsubsection{\texorpdfstring{\textbf{4. Hypothesis
Testing}}{4. Hypothesis Testing}}\label{hypothesis-testing}

\begin{itemize}
\tightlist
\item
  \textbf{T-Test}: Tests the significance of individual coefficients.
\item
  \textbf{F-Test}: Tests the joint significance of multiple
  coefficients.
\item
  \textbf{Likelihood Ratio Test}: Used for nested model comparison.
\end{itemize}

\subsection{\texorpdfstring{\textbf{Summary}}{Summary}}\label{summary}

\begin{itemize}
\tightlist
\item
  \textbf{Relaxation of Assumptions} can cause bias (e.g., endogeneity
  leads to biased \(\beta\)) or inefficiency (e.g., autocorrelation
  affects standard errors).
\item
  \textbf{Tests} help identify violations of key assumptions, and
  remedies such as using robust standard errors or instrumental
  variables can address these issues.
\end{itemize}




\end{document}
